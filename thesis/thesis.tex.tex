% Created 2023-05-21 dom 19:23
% Intended LaTeX compiler: pdflatex
\documentclass[11pt,a4paper]{article}
    \usepackage[utf8]{inputenc}
    \usepackage[T1]{fontenc}
    \usepackage{fixltx2e}
    \usepackage{graphicx}
    \usepackage{longtable}
    \usepackage{float}
    \usepackage{wrapfig}
    \usepackage{rotating}
    \usepackage[normalem]{ulem}
    \usepackage{amsmath}
    \usepackage{textcomp}
    \usepackage{marvosym}
    \usepackage{wasysym}
    \usepackage{amssymb}
    \usepackage{hyperref}
    \usepackage{mathpazo}
    \usepackage{color}
    \usepackage{enumerate}
    \definecolor{bg}{rgb}{0.95,0.95,0.95}
    \tolerance=1000
                    
    \linespread{1.1}
    \hypersetup{pdfborder=0 0 0}
\author{Gabriele Miotti}
\date{\today}
\title{}
\hypersetup{
 pdfauthor={Gabriele Miotti},
 pdftitle={},
 pdfkeywords={},
 pdfsubject={},
 pdfcreator={Emacs 28.2 (Org mode 9.6)}, 
 pdflang={English}}
\begin{document}

\tableofcontents

\documentclass{article}
\usepackage{graphicx}

\usepackage{subfiles} \% Best loaded last in the preamble

\title{The Importance of a Plaform-Agnosting Bytecode for a Decentralized Network}
\author{Gabriele Miotti}
\date{May 2023}

\begin{document}

\maketitle

\begin{abstract}

How can a distributed network agree on the execution of an arbitrary code in an trust-free way? Currently there is plenty of solution to achieve that, every solution differs by the network's structure and procols but the common ground is the presence of a bytecode that is able to run arbitrary code in a deterministic manner. The following 'paper' will describe the basic characteristics of an Platform Agnostic Bytecode and how is used in Polkadot, a distribuited network that `aims to provide a scalable and interoperable framework for multiple chains with pooled security`(polkadot-overview paper).

\end{abstract}

% $\tableofcontents IDK, it does not work

\subfile{sections/pab}

\subfile{sections/wasm}

% Should I make a little introduction on Polkadot?
% \section{What's a decentralized network?}
% \subsection{Definiotion} % remark the agreement part
% \subsection{Possible networks implementation}
% \subsection{Blockchain} % really fast explanation

\subfile{sections/polkadot}

\subfile{sections/rivals}


\end{document}
\end{document}
