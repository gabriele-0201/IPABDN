\documentclass[../main.tex]{subfiles}

\begin{document}

\section{WASM Rivals}

WASM is not the only PAB used in the blockchain space, other technologies use different solutions involving different protocols and algorithms but more important the different PABs. Examples are Ethereum with the custom PAB executed by the EVM (Ethereum Virtual Machine) or Solana that decided to used eBPF to implement the SmartConctract feature.

\subsection{EVM}

Ethereum Virtual Machine's bytecode is one of the first PAB used in blockchain, it does not follow the features used to describe a perfect PAB, it was created to be a perfect blockchain's bytecode and the Ethereum Virtual Machine is the glue that makes it executable on every machine.

The EVM is the building block of all the Etherum stack, it executes stack based code and manage all the memory and access to the storage, follows therefore the same principles of an embedder for code wasm with many features tied to the measurement of the computation onchain, called gas.

\subsection{eBPF}

\subsubsection{What is eBPF}

\href{https://ebpf.io/what-is-ebpf/}{what-is-ebpf}

eBPF stands for 'Extended Berkeley Packet Filter', what is now eBPF makes the acronym an incorrect representation of it.

Linux brought eBPF into the kernel, enabling sandboxed programs to run inside a privileged context (OS level). For lot of different reasons keeping the kernel upgraded was a difficult task and eBPF intend to solve this problem.

How can a new feature be developed once and be added to the Linux kernel? Keeping in mind that running arbitrary code developed by whoever in the kernel is absolutely not safe and the same code must be able to run on different architectures?

The operating system guarantee efficiency ans security through a JIT compiler and a verification engine for every eBPF program. To achieve that every kernel contains an eBPF VM able to checks the termination of the problem and the security guarantees.

Main points of eBPF to make the verification process possible are:
\begin{enumerate}
  \item There are no functions in the code, there is only an unique blanket of code
  \item Limited control flow
  \item Loops need to be statically defined, they are unrolled at compilation time
  \item The execution can't pass twice on the same code
\end{enumerate}

\subsubsection{Solana eBPF}

\href{https://forum.polkadot.network/t/ebpf-contracts-hackathon/1084}{solana in polkadot}

Not every program can compile to eBPF but a distributed systems need to execute arbitrary code and limitation as strict as normal eBPF are too much, Solana then (a blockchain) forked the eBPF backend of LLVM and removed lots of constraint, keeping the finality guarantee by the standard gas metering.

Solana also create a new virtual machine for eBPF, \href{https://github.com/qmonnet/rbpf}, able to check, compile and execute the code on the blockchain.

eBPF is a perfect PAB for some use cases, as the linux kernel, but it does not seem to be a good fit for blockchains, examples are:
\begin{enumerate}
  \item limited control flow
  \item limited loops
  \item 64 bit usage implies lot of checks on the memory access
  \item 8 bytes instructions are too long
\end{enumerate}


\subsection{RISC-V}

\href{https://github.com/riscv/riscv-isa-manual/releases/download/Ratified-IMAFDQC/riscv-spec-20191213.pdf}{spec-1}

RISC-V is a new instruction-set architecture (ISA), even if it is an actual bytecode for a specific hardware and seems to go against what has been describe so far as PAB there are running experiments that make this as a valid option as PAB.

There are several reasons why RISC-V could be a good PAB even if it was designed to be an ISA:
\begin{description}[style=nextline]
  \item[Real ISA suitable for direct native hardware implementation]
        RISC-V is designed to be executed on real hardware, no assumption is made on the hardware itself, the only required thing is to respect all the constraint specified in the specifications. The ISA is developed following general machine patterns with something completely new but still very related to other machine making really easy the translation.
  \item[RISC]
        The name RISC stands for Reduce Instruction Set Computer, the spec are then very small and simple making possible the creation of a base executor in really few lines of code.
  \item[Completely open ISA]
        RISC-V is an open standard, this makes possible to everyone to know the behavior and possibly create custom hardware to execute it.
  \item[ISA separated into a small base integer ISA]
        RISC-V is divided into smaller ISA, each one can work alone and can also be composed to achieve different functionalities. One of the most important division is the 32-bit and 64-bit address space division, it makes possible the creation of a sandboxed environment in ad easy and effective way.
\end{description}

\subsubsection{RISC-V for SmartConctract}

One experiment to port RISC-V in the Polkadot ecosystem as language for SmartConctract is \href{https://github.com/koute/polkavm-experiment}{polkavm-experiment}, and the discussion is reported in this \href{https://forum.polkadot.network/t/exploring-alternatives-to-wasm-for-smart-contracts/2434}{forum}.

The spec are so easy that creating an interpreter required only one day and the JIT only two days more. Those two executor were then tested against other PAB with different executors, every one executed the same code: a NES emulator written in Rust and the interpretation was how close to 60 FPS the code can get.

The results are:
\begin{itemize}
    \item wasmi: 108ms/frame (~9.2 FPS)
    \item wasmer singlepass: 10.8ms/frame (~92 FPS)
    \item wasmer cranelift: 4.8ms/frame (~208 FPS)
    \item wasmtime: 5.3ms/frame (~188 FPS)
    \item solana\_rbpf (interpreted): 6930ms/frame (~0.14 FPS)
    \item solana\_rbpf (JIT): ~625ms/frame (~1.6 FPS)
    \item RISC-V interpreter: ~800ms/frame (1.25 FPS)
    \item RISC-V JIT: ~25ms/frame (~40 FPS)
\end{itemize}

As you can see the simplicity of RISC-V made possible the creation of a JIT compiler able to be faster than lot of competitive compiler already being used in different realities.

An important difference rather then eBPF are dedicated instructions for host functions (syscall) and the support in rustc and LLVM. One of the main constraint is the number of register used in RISC-V, 32 registers while most architectures use 16, this ask to the compiler or interpreter to properly manage those extra registers.

\end{document}
