\documentclass[../main.tex]{subfiles}
\graphicspath{{\subfix{../images/}}}

\begin{document}

\section{WASM}
\subsection{Definitions}
(\href{https://webassembly.github.io/spec/core/}{wasm-core})
(\href{https://wiki.polkadot.network/docs/learn-wasm}{wasm-polkadot})
(\href{https://www.ngzhian.com/relaxed-simd/core/_download/WebAssembly.pdf}{wasm-spec})

WebAssembly, shortened to Wasm, is a binary instruction format for a stack-based virtual machine(wasm-core). It is a platform-agnostic binary format, meaning it will run the exact instructions across whatever machine it operates on(wasm-polkadot).

`asm.js` was Wasm precursor, but browser vendors like Mozilla, Google, Microsoft, and Apple focus on the Wasm designed. The main goal was to create a binary format where a couple of the main wanted features were: compact, support for streaming complication and sanboxed execution.

Wasm is currently a compilation target for a lot of high level language, this allow languages to enter in the web-world, in client or server applications, but also in completely different application. Examples are plugins, if an application is able to accept, execute and make in interact the application itself then we have a entire set of High-Level languages able to create those plugins having wasm as minumum common divisor. Then we have a entire set of High-Level languages able to create those plugins having wasm as minimum common divisor.

The main design goals in the wasm specification introduction (wasm-core) are:
\begin{description} [style=nextline]
  \item[Fast] It's design allow to create executor with so less overhead that the execution is almost fast as native code
  \item[Safe] It is completely memory-safe as long as the executor is correctly behaving, sandboxing the execution properly
  \item[Well-defined] The definition of the binary format makes easy to create a valid executor that makes the code behave correctly
  \item[Hardware-independent] The compilation process is independent by the architecture that will run the code
  \item[Language-independent] There's no strong influence by other binary format, language or programming model
  \item[Platform-independent] It can be compiled and be executed on all modern architectures, embedded systems or applications as browsers
  \item[Open] There is simple way to interoperate the with executor/environment
\end{description}

Other important consideration are made on the efficiency and portability, the word used to described those two features are: compact, modular, efficient, streamable, and parallelizable.
Something not clear at first look is \"modular\", it means that the program can be split into smaller parts and those can be transmitted, cahed or consumed separately.

In the following chapters the words executor, embedder or environment have the same meaning.

\subsection{Specifications}

The specification does not make any assumption on the environment, this makes it completely contraint-less, it just must follows all the defined  instruction set, binary encoding, validation, and execution semantics.(wasm-core)

Wasm is stack-based, this means that the instruction set is very different from the standards architecture's bytecode that normally are registered-based. Wasm has also a one-to-one text representation other than the normal binary representation, of course it makes the code less compact but almost human readable.

All the concepts present in the specifications are very high-level even if it is a low level language, those concepts are the following:

\begin{description} [style=nextline]
  \item[Values]
        Wasm has only four data type, integers and foaloating points (following IEEE 754 standard) both 32 and 64 bits
  \item[Instructions]
        Being a stack based language every instruction works implicit on a stack but there is a general division between:
        \begin{itemize}
          \item Simple Instructions, performing basic operations on data
          \item Control Flow, allowing to follow some high-level language control flow having nested blocks
        \end{itemize}
  \item[Traps]
        Those are instructions which immediately aborts the execution, the termination is not endled by wams itself but by the embedder
  \item[Functions]
        Being so new, this assemply-like language, allow users to work with functions abstracting some standards assemply's complexity
  \item[Table] NOPE
  \item[Linear Memory]
        This is where the communication between the code and the environment happens, like the name says this is a contiguous area of memory given to the code. This memory is very crucial for the security considerations that we will see later.
  \item[Modules]
        A Module contains everything just explained, this is the logical container of the code. Every wasm code is made by a single module.
  \item[Imports] TODO
  \item[Exports] TODO
  \item[Embedder]
        Of course to be executed wasm needs the embedder, the main jobs are:
        \begin{itemize}
          \item loading and initiate a new module
          \item provide imports
          \item manage exports
        \end{itemize}
\end{description}

Other important concepts explained in the specification are wasm phases, they are:

\begin{description} [style=nextline]
  \item[Decoding]
        Decode the binary format to the specified abstract syntax, the implementation could also compile directly to machine code.
  \item[Validation]
        A decoded module has to be valid, the validation consists in check a set of well-formedness conditions to guarantee that the module is meaningful and safe (wasm-core, un po' troppo copiato questo)
  \item[Execution]
        \begin{itemize}
          \item Instantiation, set up state and execution stack of a module
          \item Invocation, calling a function provided by the module to start the effective execution
        \end{itemize}
\end{description}

\subsection{Execution}
(\href{https://docs.wasmtime.dev/}{wasmtime})

Wasm specifications can be perfect making everything unbreakable but at the end everything depends on the embedder's implementation, if it is not secure then wasm execution itself is not. Wasm can be executed in different ways, the main one used in the blockchain world are: Ahead Of Time Complication (AOT), Just In Time Compilation (JIT), Single Pass Compilation and Interpretation.

Every type has its own advantages but also requires different tricks to make everything secure, one important thing provided by wasm is an intensive test suite to check the correctness of the embedder, all the tests are maintained here: \href{https://github.com/WebAssembly/spec/tree/main/test/core}.

The common divisor for the first three types of execution is the transpilation of a stack-based bytecode to a register one, this means that the compiler tries to elide every access to the main stack used by wasm allocating everything needed in the registers. It's impossible to completely avoid the interaction with the stack, this means that at the end the final bytecode will use a stack, in some cases the stack of the embedder. (This is could be explained more in depth but requires a lot of time to make sure everything is correct)

There is though an important difference between value stack and shadow stack ... (it's worth to explain?)

\subsubsection{AOT}

AOT is the standard compilation, all the code is compiled and later executed. Wasmtime is a wasm embedder, it is a stand alone wasm environment but it could be also used as library to create a wasm environment in your bigger application. Wasmtime offer offers this feature, it accept wasm in text or binary format and compiles it to some architecture's bytecode.

\subsubsection{JIT}

JIT is a dynamic compilation where the bytecode si compiled only if it needed, the compiler first need to create an intermediate representation to being able to compile the different parts only if the execution requires to. A really simple example to make is: we have a program that given an input calls function A or B, the JIT then will understand this structure and compile the entry point and only one function between A or B based on the initial input.

JIT is really efficient because the huge work is already been done by the first phase of compilation where the High-Level language is compiled into wasm, now the JIT has only to compensate all the differences between bytecose, but they are both really low level and this makes this process really more efficient then the first one.

Wasmtime is specialized in this type of execution and it makes it really efficient keeping every secure.

\subsubsection{SP} %Single Pass Compile%

A Single Pass compiler is a restriction of AOT compiler, the complexity of the compilation must be O(n) so the wasm bytecode will be scanned through only once. Like every other compilation method here the trade off si not to create efficient final code but to create the final bytecode as fast as possible, why not optimizing should be worth? Because wasm is already compiled from some higher level language and the compiler probabily alredy did a lot of optimizations.

Wasmer is wasm embedder with a lot of features, in particular they implemented a single pass compiler for all the most important architecures.

\subsubsection{Interpret}
\href{https://github.com/paritytech/Wasmi}{wasmi-spec}

Interpretation is easier to think way to execute wasm, it becomes like any other interpreted language executed by a specialized Virtual Machine. There are multiple ways to interpret code but we will focus on one of the most efficient wasm interpreter, wasmi.

Wasmi is an efficient WebAssembly interpreter with low-overhead and support for embedded environment such as WebAssembly itself (wasmi-spec).

Currenly the first wasm bytecode pass produce another stack based bytecode, called WASMI IR, and then this bytecode is interpreted by the Virtual Machine, even with this transpilation it is only 5 time slower then the compilation to native bytecode of the architecture. (Resource to be found)

\subsection{Security guarantee}

Wasm's aim is to be extremely secure the the specifications describe a lot of aspects to achieve that. The security guarantee depends mostly on the execution, WebAssembly is designed to be translated into machine code running directly on the host’s hardware. Being so portable wasm can be sent to someone and be executed freely, examples in every browsers. We are running wasm our machines every day and if would not be so secure then we would had notice a lot of problems.

Executing wasm is potentially vulnerable to side channel attacks on the hardware level(wasm-core) and isolation is the only way to make secure the execution. If the embedder translate one on one every instructions then everything can be computed on your computer, but nothing dangerous if the code has no access to the environment where is executed.

The problem is that a completely isolation makes wasm useless, so there's a way to communicate with the environment or also have access to it, but those features are extremely limited and designed to be secure.

\subsubsection{Linear Memory}
(\href{https://hacks.mozilla.org/2017/07/memory-in-webassembly-and-why-its-safer-than-you-think/}{linear-memory})

From WebAssemply you have direct access to raw bytes, but where are allocated those bytes? Wasm uses a MemoryObject provided by the embedder to describe the only accessible memory, beside the stack. (linear-memory)

Wasm does not have pointer types, values in the linear memory ara accessed as a vector, where the first index of the memory is 0.

Wasm, for security reason that will be explained in next chapters, works in a 32-bit address space, this makes usable only 4GiB of memory. Being the position of the Linear Memory memory unknown to the wasm blob every load or store the the memory is made passing through the embedder that will also do bounds checks to make sure the address is inside the wasm Linear Memory.

This level of control makes impossible to have memory leak in the environment during the wasm execution because there is a completely memory isolation. (linear-memory)

% \subsubsection{Stack}
% \href{https://hackmd.io/@pepyakin/SkmPKGhiq}{stack more or lesse expained}
%
% It really depends on the implementation of the execution
%
% We explained later that wasm is sjj
%
% How the stack is managed?
%
% substrate -> wasm-instrument -> stack injection to make it deterministic (shadow stack at the beginning)

\subsubsection{Communication in a sandboxed environment}

We just described how wasm provides no ambient access to the computing environment in which code is executed (wams-core), thanks to a mix of wasm design choice and embedder implementation. But how works then the interaction with the environment?

\

Every interaction can be done by a set of functions provided by the embedder and imported in the Wasm module(wasm-core), those functions are called \textbf{Host Functions}. Host functions allow wasm code to access to resources, operating system calls or any other type of computation offered by the embedder.

\subsubsection{Wasmtime Security guarantee}
\href{https://docs.wasmtime.dev/}{wasmtime-book}

Wasmtime is widely used in different environmets and one of those is all the polkadot ecosystem, precisely wasmtime is used inside substrate. Substrate is a framework to develop blockchains that with some tweaks can become parachains, polkadot itself is a substrate based chain.

Wasmtime main goals is to execute untrusted code in a safe manner.(wasmtime-book)

Some features that makes executing wasm by wasmtime so secure are just inherith by wasm specifications, some examples are: the callstack is inaccessible, pointers are compiled to offsets into linear memory, there's no undefined behavior and every interaction with the outside world is done through imported and exported functions(wasmtime-book).

Wasmtime to those features adds a lot of mitigations to limit issues:
\begin{itemize}
  \item Linear memories by default are preceded with a 2GB guard region
  \item Wasmtime will zero memory used by a WebAssembly instance after it's finished.
  \item Wasmtime uses explicit checks to determine if a WebAssembly function should be considered to stack overflow (this is really a deep concepts about how wasmt time manage the wasm's stacks, not sure if it's worth to keep)
  \item The implementation language of wasmtime, Rust, helps catch mistakes when writing Wasmtime itself at compile time
\end{itemize}

\end{document}
