\documentclass[../main.tex]{subfiles}

\begin{document}

\section{Platform-Agnostic Bytecode}
\subsection{Definition}

A Platform-Agnostic Bytecode (PAB) is
%LB>> it this a standard term? where is it defined? or is it your own term? please specify
a bytecode that follows those two main principles:

\begin{itemize}
    \item Turing Completeness
    \item Support for tooling that makes it executable on every machine
\end{itemize}

A bytecode like this ideally is designed to be executed on a virtual machine that follows general patterns. This design should make easier the compilation to another real machine's bytecode. Examples of real architectures with specified bytecode are AMD and Intel with x86 or ARM with aarch64. % TODO: check this last sentence

\subsection{Execution}

PABs require multiple phases of compilation. The first one is encountered when you want to compile your High-Level language to the PAB using a Cross-Compiler. Once you have the PAB code, you should be able to run it on every machine using another compiler that will create the final executable code.

Re-compiling is not the only way to execute a PAB, another common solution is to implement a Virtual Machine (VM) able to run arbitrary PAB code interpreting it.

\subsection{Key features}

Every bytecode, ideally, can become a PAB if tools to make it runnable to different machine exist. There are however some metrics to define which one is better than others; example of metrics are:

\begin{description}[style=nextline]
  \item[Hardware Independence]
        A bytecode can't be a PAB if tightly related to specific hardware. A PAB can be defined as such if there is no direct connection between bytecode and hardware, the only exception is if the relations require only a little overhead for the execution on different hardware.
%LB>> unclear to me: do you mean 'It is however tolerated that the PAP requires some minor adaptations to run on differnet hardware'?

  \item[Sandboxing]
        The machine used to execute the PAB is defined as \textit{embedder}. The embedder will execute arbitrary code, possibly malicious.  A sandboxed environment is the typical solution to overcome any security problem. The execution in a different environment makes almost impossible to compromise the embedder from the PAB code. Implemening a proper sandboxed environment is embedder dependent but a PAB can be more or less suitable for this feature.

        \begin{figure}[h]
          \centering
          \includegraphics[width=0.4\linewidth]{sandboxed_env.png}
          \caption{Sandboxing graphic example}
          \label{fig:Sandboxing graphic example}
        \end{figure}

  \item[Efficiency]
        The efficiency of a PAB has several facets, it could refer to:

        \begin{itemize}
          \item the efficiency of compiling High-Level Language to the PAB
          \item the efficiency of the execution of the PAB. In this case it could refer to the compilation to the final bytecode and the subsequent execution, the interpretation or more complex solutions
        \end{itemize}

        Generally the first is not really related to the PAB, but more on the used tools (examples gcc, rustc, etc.). The execution efficiency is the real deal: how fast a PAB can be executed on a machine is crucial.
  \item[Tool Simplicity]
        The easiness of compiling a High-Level language and  executing the PAB is very important to make it usable by every one.
  \item[Support as Compilation Target]
        Writing bytecode by hand (or any text representation) 

%LB>> what is "text representtation"? please clarify

        is something really rare and done only in specific cases. Every compiled language has a compiler to make this, and is very important for a PAB to support the compilation from as many languages as possible.
\end{description}

\subsection{Current usage}

PAB are already widely used. A few examples are:

\begin{itemize}
  \item
        The Java Virtual Machine (JVM) is one of the first that made the portability of the code one of the main concern of the language
  \item
        Linux brought eBPF 
%LB>> please explain the achronym

        into the kernel, enabling arbitrary programs to be executed in a privileged context (OS level)
  \item
        LLVM IR 
%LB>> please explain the achronym
        
        is the LLVM assembly language, it provides type safety, low-level operations, flexibility, and the capability of representing ‘all’ high-level languages cleanly. It is the common code representation used throughout all phases of the LLVM compilation strategy. ~\cite{LLVM}
  \item
        WebAssembly is a safe, portable, low-level code format designed for efficient execution and compact representation. ~\cite{wasm-core-spec}
\end{itemize}

\subsection{PAB in blockchains}

Blockchains are distributed systems that needs to agree on the execution of arbitrary code performed on different machines. The code execution must conclude to the same result, regardless of the machine the code is running on. What has just been described is called `deterministic execution`. PABs are the solution for both problems.
%LB>> "both" refers to?


\end{document}
