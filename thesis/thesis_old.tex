\documentclass{article}

\usepackage[letterpaper,top=2cm,bottom=2cm,left=3cm,right=3cm,marginparwidth= 2cm]{geometry}
\linespread {1.5}

\usepackage{graphicx}
\graphicspath{{\subfix{./images/}}}

\usepackage{subfiles} % Best loaded last in the preamble
\usepackage{enumitem}
\usepackage{hyperref}
\hypersetup{
    colorlinks=true,
    linkcolor=blue,
    filecolor=magenta,
    urlcolor=cyan,
    }

\title{The Importance of a Plaform-Agnosting Bytecode for a Decentralized Network}
\author{Gabriele Miotti}
\date{May 2023}

\begin{document}

\maketitle

\begin{abstract}

How can a distributed network agree on the execution of an arbitrary code in a trust-free way? Currently there are plenty of solutions to achieve that. Every solution differs by the network's structure and protocols but the common ground is the presence of a bytecode that is able to run arbitrary code in a deterministic manner. This paper will describe the basic characteristics of a Platform Agnostic Bytecode and how is used in Polkadot, a distributed network that `aims to provide a scalable and interoperable framework for multiple chains with pooled security`(polkadot-overview paper).

\end{abstract}

{
  \hypersetup{linkcolor=black}
  \tableofcontents
}

\subfile{sections/pab}

\subfile{sections/wasm}

\subfile{sections/polkadot}

\subfile{sections/rivals}

% Conclusions?

\bibliographystyle{plain}
\bibliography{refs}
\end{document}
