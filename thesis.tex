\documentclass{article}
\usepackage{graphicx}

\title{The Importance of an agnosting plaform bytecode for a decentralized network}
\author{Gabriele Miotti}
\date{May 2023}

\begin{document}

\maketitle

\begin{abstract}

How can a distributed network agree on the execution of an arbitrary code in an trust-free way? Currently there is plenty of solution to achieve that, every solution differs by the network's structure and procols but the common ground is the presence of a bytecode that is able to run arbitrary code in a deterministic manner. The following 'paper' will describe the basic characteristics of an Platform Agnostic Bytecode and how is used in Polkadot, a distribuited network that `aims to provide a scalable and interoperable framework for multiple chains with pooled security`(polkadot-overview paper).

\end{abstract}

\tableofcontents

\section{Agnostic Platform Bytecode}
\subsection{Definition of an APB}
\subsection{Key features of an APB}
\subsection{Current usage of APBs}
\subsection{APBs used in blockchains}


% \section{What's a decentralized network?}
% \subsection{Definiotion} % remark the agreement part
% \subsection{Possible networks implementation}
% \subsection{Blockchain} % really fast explanation

\section{Polkadot}
\subsection{What's Polkadot?}
\subsection{Polkadot protocol}
\subsection{APBs in Polkadot}
\subsubsection{Key features request for an APB}
\subsubsection{Usage of APB in Polkadot}
\paragraph{STF}
\paragraph{SmartContracts}
\paragraph{SPREE}

\section{WASM}
\subsection{History}
\subsection{Specifications}
\subsection{Sandboxing}
\subsection{Execution}
\subsubsection{Interpreter}
\subsubsection{Single pass Interpreter}
\subsubsection{JIT}
\subsubsection{Compilation}

\section{WASM Competitors}
\subsection{EVM}
\subsection{Solana eBPF}
\subsection{LLVM}
\subsection{eBPF}
\subsection{RISC-V}

\end{document}
